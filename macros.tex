% !TEX root = paper.tex

\algrenewcommand\algorithmicindent{0.5em}

\newcommand{\showindraft}[1]{#1}

\newcommand{\ignore}[1]{}

\newcommand{\si}[1]{\showindraft{\textcolor{purple}{\textbf{SI:} #1}}}
\newcommand{\dt}[1]{\showindraft{\textcolor{magenta}{\textbf{DT:} #1}}}

\newcommand{\ie}{i.e.,\ }
\newcommand{\eg}{e.g.,\ }
\newcommand{\etal}{et al.}
\newcommand{\etc}{etc.\xspace}
\newcommand{\SE}{SE\xspace}
\newcommand{\klee}{\text{KLEE}\xspace}
\newcommand{\llvm}{\text{LLVM}\xspace}
\newcommand{\coq}{\text{Rocq}\xspace}
\newcommand{\coqc}{\textit{coqc}\xspace}
\newcommand{\ltac}{\textit{Ltac}\xspace}
\newcommand{\vellvm}{\textit{Vellvm}\xspace}
\newcommand{\code}[1]{\texttt{#1}}
\newcommand{\sdots}{...}

\newcommand{\eqdef}{=}

\newcommand\rulenamedef[1]{\small\textsc{(#1)}}
\newcommand\rulenameref[1]{\textsc{#1}}
\newcommand\ruledef[3]{\infer[\rulenamedef{#1}]{#3}{#2}}
\newcommand\ruledefsmall[3]{\infer[\rulenamedef{#1}]{\scalebox{0.9}{$#3$}}{\scalebox{0.8}{$#2$}}}

% colors
\definecolor{stringcolor}{RGB}{0,0,0}
\definecolor{codecolor}{RGB}{33, 70, 158}
\definecolor{bvcolor}{RGB}{173, 51, 51}
\definecolor{smtcolor}{RGB}{189, 163, 0}

% Notations for domains
\newcommand\Z{\mathds{Z}}
\newcommand\Nat{\mathds{N}}
\newcommand\Bool{\mathit{Bool}}
\newcommand\Types{\mathit{T}}
\newcommand\Vals{\mathit{V}}
\newcommand\Ids{\mathit{Id}}
\newcommand\Locs{\mathit{L}}
\newcommand\Funcs{\mathit{F}}
\newcommand\Stores{\Sigma}
\newcommand\Exprs{\mathit{E}}
\newcommand\MemoryObjects{\mathit{MO}}
\newcommand\Heaps{\mathit{H}}
\newcommand\Intervals{\mathit{I}}

% LLVM
\newcommand{\nameid}[1]{\mathit{\textcolor{stringcolor}{#1}}}
\newcommand{\pointertypeof}[1]{\mathit{#1{*}}}
\newcommand{\arraytype}[2]{\mathit{[#1 \times #2]}}
\newcommand{\structtype}[2]{\mathit{struct}\{#1_1, #1_2, \sdots, #1_n\}}
\newcommand\tid{\mathit{id}}
\newcommand\tbinop{\mathit{binop}}
\newcommand\tcmpop{\mathit{cmpop}}
\newcommand\toper{\mathit{op}}
\newcommand\tcast{\mathit{cast}}
\newcommand\tinstr{\mathit{instr}}
\newcommand\tfunc{\mathit{func}}
\newcommand\tfuncs{\mathit{funcs}}
\newcommand\tblk{\mathit{blk}}
\newcommand\tmodule{\mathit{module}}
\newcommand\tif{\mathrm{if}}
\newcommand\opcode[1]{\textsf{#1}}
\newcommand\tbr{\opcode{br}}
\newcommand\tcall{\opcode{call}}
\newcommand\tphi{\opcode{phi}}
\newcommand\tret{\opcode{ret}}
\newcommand\tload{\opcode{load}}
\newcommand\tstore{\opcode{store}}
\newcommand\tgep{\opcode{getelementptr}}
\newcommand\tdef{\opcode{def}}
\newcommand\tunreachable{\opcode{unreachable}}
\newcommand\tundef{\mathit{undef}}
\newcommand\tpoison{\mathit{poison}}
\newcommand{\createinst}{\mathit{inst}}

\newcommand{\funcid}{\mathit{id}}
\newcommand{\funcentry}{\mathit{entry}}
\newcommand{\funcargs}{\mathit{args}}

\newcommand{\module}{\Delta}
\newcommand\locsep{{\hspace{1pt}:\hspace{1pt}}}
\newcommand\codequote[2][\textcolor{codecolor}]{{#1{\texttt{`}{#2}\texttt{'}}}}
\newcommand\instrat[1]{\module[{#1}]}
\newcommand\nextinstr[1]{{#1}{+}1}
\newcommand\intrinsicfunc[1]{\textit{#1}}

\newcommand{\mapaccess}[2]{#1(#2)}
\newcommand{\sem}[1]{\llbracket #1 \rrbracket}
\newcommand\evalexprty[3]{\sem{#1}^{#3}{#2}}
\newcommand\valoftype[2]{\textcolor{bvcolor}{{#1}_{#2}}}
\newcommand\bvsort[1]{\textsf{i}{\textsf{#1}}}
\newcommand\bvval[2]{\valoftype{#1}{\bvsort{#2}}}

\newcommand\welldef[1]{#1}

\newcommand{\location}{\ell}
\newcommand{\localstore}{\sigma}
\newcommand{\stack}{\mathit{\kappa}}
\newcommand{\memory}{\mathit{\mu}}
\newcommand{\initstate}{\textit{init}}
\newcommand{\createlocation}[3]{#1\locsep#2\locsep#3}
\newcommand{\stackframe}[3]{\langle{#1},{#2},{#3}\rangle}
\newcommand{\progstate}[4]{\langle{#1},{#2},{#3},{#4}\rangle}
\newcommand{\step}{\rightarrow}
\newcommand{\multistep}{\mathrel{{\step}{}^*\!}}
\newcommand{\errorstate}{\mathit{error}}
\newcommand{\issafe}{\mathit{safe}}
\newcommand{\nsstep}{\stackrel{\hspace{-2pt}\raisebox{1pt}{\tiny{\textit{ns}}}}{\rightarrow}}
\newcommand{\multinsstep}{\mathrel{{\nsstep}{}^*\!}}
\newcommand{\issafens}{\mathit{safe_{ns}}}
\newcommand{\pointerwidth}{\textsf{ptr}}
\newcommand{\indexwidth}{\textsf{idx}}
\newcommand{\pointertype}{\bvsort{\pointerwidth}}
\newcommand{\indextype}{\bvsort{\indexwidth}}
\newcommand{\addr}{b}
\newcommand{\size}{s}
\newcommand{\arrayvalue}{a}
\newcommand{\mo}{\mathit{mo}}
\newcommand{\mobj}[3]{\langle{#1},{#2},{#3}\rangle}
\newcommand{\heaprepr}[2]{\langle{#1},{#2}\rangle}
\newcommand{\zeroext}{\mathit{zeroext}}
\newcommand{\signext}{\mathit{signext}}
\newcommand{\convert}{\mathit{convert}}
\newcommand{\sizeof}[1]{{\texttt{sizeof(}{#1}\texttt{)}}}
%\newcommand{\isbaseaddr}[2]{#1 \rightsquigarrow #2}
\newcommand{\isbaseaddr}[2]{#2 = \mobj{#1}{\_}{\_}}
%\newcommand{\isinbounds}[3]{\mathit{is\_inbounds}(#1, #2 #3)}
\newcommand{\isinbounds}[3]{#1 \overset{#2}{\rightsquigarrow} #3}
\newcommand{\invalidpointer}{\mathit{invalid\textit{-}pointer}}
\newcommand{\invalidbaseaddress}{\mathit{invalid\textit{-}base\textit{-}address}}
\newcommand{\allocate}{\mathit{allocate}}
\newcommand{\replace}{\mathit{replace}}
\newcommand{\readmo}{\mathit{read}}
\newcommand{\writemo}{\mathit{write}}
\newcommand{\evalgep}{\mathit{eval\textit{-}gep}}

\newcommand\moWrite[4]{#1[#2\overset{#3}{\mapsto}#4]}
\newcommand\moRead[3]{#1[#2]^{#3}}
\newcommand\heapRpl[3]{#1[#3/#2]}
\newcommand\heapWrite[5]{\heapRpl{#1}{#2}{\moWrite{#2}{#3}{#4}{#5}}}

% SMT
\newcommand{\true}{\mathit{true}}
\newcommand{\false}{\mathit{false}}
\newcommand{\sat}{\mathit{sat}}
\newcommand{\unsat}{\mathit{unsat}}
\newcommand{\transform}{\mathit{transform}}
\newcommand{\subexpr}{\sqsubseteq}
\newcommand{\evalmodel}[2]{#1[#2]}
\newcommand{\smtsort}[1]{\textsf{bv}_{#1}}
\newcommand{\bvconst}[2]{\textcolor{smtcolor}{#2_{#1}}}
\newcommand{\bvadd}{\textsf{bvadd}}
\newcommand{\bvsub}{\textsf{bvsub}}
\newcommand{\bvmul}{\textsf{bvmul}}
\newcommand{\bvzext}{\textsf{bvzext}}
\newcommand{\bvsext}{\textsf{bvsext}}
\newcommand{\bvextract}{\textsf{bvextract}}
\newcommand{\arraysort}[2]{Array\langle#1,~#2\rangle}
\newcommand{\arrayconst}[1]{K(#1)}
\newcommand{\arrayselect}[2]{#1[#2]}
\newcommand{\arraystore}[3]{#1[#2 \mapsto #3]}
\newcommand{\arrayinitvalue}{\bvconst{8}{0}}

% symbolic semantics
\makeatletter
\makeatother
\newcommand{\SymStores}{\tilde\Sigma}
\newcommand{\SymMemoryObjects}{\widetilde{\mathit{MO}}}
\newcommand{\symlocalstore}{\tilde\sigma}
\newcommand{\symstack}{\tilde{\kappa}}
\newcommand{\pc}{\varphi}
\newcommand{\symstate}[5]{\langle{#1},{#2},{#3},{#4},{#5}\rangle}
\newcommand{\errorsymstate}{\mathit{error}}
\newcommand{\symvar}{\alpha}
\newcommand{\overapprox}{\triangleright}
\newcommand{\overapproxvia}[1]{\triangleright_{#1}}
\newcommand{\initsymstate}{\widetilde{\textit{init}}}
\newcommand{\symstep}{\overset{\hspace{-2pt}\sim}{\rightarrow}}
\newcommand{\symmultistep}{\mathrel{{\symstep}{}^*\!}}
\newcommand{\infeasible}{\mathit{infeasible}}
\newcommand{\symstackframe}[3]{\langle{#1},{#2},{#3}\rangle}
\newcommand{\freevars}{\mathit{FV}}
\newcommand{\smtarray}{\tilde{a}}
\newcommand{\symmo}{\widetilde{\mathit{mo}}}
\newcommand{\symmemory}{\tilde{\mu}}
%\newcommand{\symisinbounds}[4]{\mathit{is\_inbounds}(#1, #2, #3, #4)}
\newcommand{\symisinbounds}[4]{#4[#1] \overset{#2}\rightsquigarrow #3}
%\newcommand{\symisbaseaddr}[3]{#3[#1] \rightsquigarrow #2}
\newcommand{\symisbaseaddr}[3]{#3[#1] \rightsquigarrow #2}
\newcommand{\baseaddrcond}{\mathit{bc}}
\newcommand{\rangecond}{\mathit{rc}}
\newcommand{\syminvalidpointer}{\mathit{invalid\_pointer}}
\newcommand{\syminvalidbaseaddress}{\mathit{invalid\_base\_address}}
\newcommand{\overflowcond}{\mathit{overflow}}

% execution tree
\newcommand{\equivstate}{\equiv}
\newcommand{\safeet}{\textit{safe-et}}
\newcommand{\leaf}{\mathit{leaf}}
\newcommand{\tree}{\mathit{tree}}
\newcommand{\attrstate}{\mathit{state}}
\newcommand{\attrchildren}{\mathit{children}}
\newcommand{\repr}{\mathit{repr}}

% evaluation
\newcommand{\base}{\textit{Base}\xspace}
\newcommand{\proofopt}{\textit{PL}\xspace}
\newcommand{\oom}{\textit{OOM}}
\newcommand{\libtasn}{\textit{libtasn1}\xspace}
\newcommand{\libosip}{\textit{libosip}\xspace}
\newcommand{\coreutils}{\textit{GNU coreutils}\xspace}
\newcommand{\svcomp}{\textit{SVComp}\xspace}

\newcommand\x{{\small$\times$}\xspace}

\lstdefinelanguage{coq}{
mathescape=true,
texcl=false,
morekeywords=[1]{Section, Module, End, Require, Import, Export,
  Variable, Variables, Parameter, Parameters, Axiom, Hypothesis,
  Hypotheses, Notation, Local, Tactic, Reserved, Scope, Open, Close,
  Bind, Delimit, Definition, Let, Ltac, Fixpoint, CoFixpoint, Add,
  Morphism, Relation, Implicit, Arguments, Unset, Contextual,
  Strict, Prenex, Implicits, Inductive, CoInductive, Record,
  Structure, Canonical, Coercion, Context, Class, Global, Instance,
  Program, Infix, Theorem, Lemma, Corollary, Proposition, Fact,
  Remark, Example, Proof, Goal, Save, Qed, Defined, Hint, Resolve,
  Rewrite, View, Search, Show, Print, Printing, All, Eval, Check,
  Projections, inside, outside, Def},
morekeywords=[2]{forall, exists, exists2, fun, fix, cofix, struct,
  match, with, end, as, in, return, let, if, is, then, else, for, of,
  nosimpl, when},
morekeywords=[3]{Type, Prop, Set, true, false, option},
morekeywords=[4]{pose, set, move, case, elim, apply, clear, hnf,
  intro, intros, generalize, rename, pattern, after, destruct,
  induction, using, refine, inversion, injection, rewrite, congr,
  unlock, compute, ring, field, fourier, replace, fold, unfold,
  change, cutrewrite, simpl, have, suff, wlog, suffices, without,
  loss, nat_norm, assert, cut, trivial, revert, bool_congr, nat_congr,
  symmetry, transitivity, auto, split, subst, left, right, autorewrite},
morekeywords=[5]{by, done, exact, reflexivity, tauto, romega, omega,
  assumption, solve, contradiction, discriminate},
morekeywords=[6]{do, last, first, try, idtac, repeat},
morecomment=[s]{(*}{*)},
showstringspaces=false,
tabsize=2,
extendedchars=false,
sensitive=true,
breaklines=false,
captionpos=b,
columns=[l]flexible,
}[keywords,comments,strings]
