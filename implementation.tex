\section{Implementation}

% TODO: prototype?
% TODO: we fixed the issues in our version

Our prototype implementation consists of two components:
A formalization of the concrete and symbolic semantics (\Cref{sec:concrete-semantics,sec:symbolic-semantics}) and an extension of the \SE engine.
Our formalization is manually written in \coq (8.19.1) and consists of roughly TBD lines of code, excluding third-party dependencies.
Our proof-generation approach (\Cref{sec:proof-generation}) is implemented on top of \klee~\cite{klee},
a \emph{state-of-the-art} \SE engine that operates on \llvm bitcode~\cite{llvm}.
During the implementation of our prototype,
we found several implementation issues in \klee that relate to memory modeling.
We now give a detailed overview of these issues.

\subsection{Implementation Issues in \klee}
% TODO: mention that different parts assume different index types
To represent the contents of a memory object, \klee uses SMT arrays of sort:
\[ \smtsort{32} \rightarrow \smtsort{8} \]
Based on this, we conclude that the index type is assumed to be \bvsort{32}.

\subsubsection{Incorrect Interpretation of GEP}
The algorithm used in \klee to handle the $\tgep$ instruction is given in~\Cref{alg:gep-klee}.
As can be seen,
the offset computed at each iteration is converted to an expression of sort $\smtsort{64}$,
while the index type is assumed to be \bvsort{32}.
Even if we assume that the index type is \bvsort{64},
the algorithm does not check overflows when computing the resulting pointer.

\begin{algorithm}[t]
\caption{algorithm for handling \tgep in \klee.}
\label{alg:gep-klee}
\begin{algorithmic}[1]

\Function{compute-pointer}{$...$}
\EndFunction

\end{algorithmic}
\end{algorithm}

\subsubsection{Allocation}
According to the \llvm semantics,
if the allocation size exceeds the largest \emph{signed integer} that fits into the index type,
then the result is undefined behavior.
However, \klee currently does not check this condition.
In fact, \klee silently truncates the allocation size to the \code{uint32\_t} type, without reporting any error to the user.
To illustrate this behavior, consider the code snippet from~\Cref{figure:issue-alloc}.
When executing~\cref{line:alloc-overflow},
the allocation size (\code{UINT32\_MAX + 1 + 10}) is internally casted to the \code{uint32\_t} type,
so the actual allocation size is 10.

We found a similar issue in the implementation of $\tload$ and $\tstore$.
When a memory object is accessed,
the SMT expression corresponding to the offset is always truncated to an SMT expression of sort BV32.
As a result, when \code{p} is accessed at~\cref{line:alloc-store}, the computed offset is the SMT expression 0.
% which means that...

\begin{figure}[t]
\lstinputlisting[basicstyle=\scriptsize\ttfamily,language=c,mathescape=true,breaklines=false]{code/alloc.c}
\caption{...}
\label{figure:issue-alloc}
\end{figure}

%Discrepancies related to the index type: ... \\
%interger overflow when setting the size of a memory object (in KLEE's representation): ...
